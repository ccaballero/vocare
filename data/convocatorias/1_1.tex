\documentclass[letterpaper,11pt]{article}

\usepackage[spanish]{babel}
\usepackage[utf8]{inputenc}
\usepackage[left=2.5cm,top=2cm,right=2.5cm,bottom=2cm]{geometry}
\usepackage{enumerate}
\usepackage{multirow}
\usepackage{url}

\title{CONVOCATORIA A CONCURSO DE MÉRITOS Y PRUEBAS DE CONOCIMIENTOS PARA OPTAR A AUXILIATURAS EN LABORATORIO DE COMPUTACIÓN, DE MANTENIMIENTO Y DESARROLLO\\GESTIÓN I/2012 - II/2012\\}
\date{}
\begin{document}

\maketitle
El Departamento de Informática y Sistemas junto a las Carreras de Ing. Informática e Ing. de Sistemas, de la Facultad de Ciencias y Tecnología, convoca al concurso de meritos y examen de competencia para la provisión de Auxiliares del Departamento, tomando como base los requerimientos que se tienen programados para la gestión I/2012 - II/2012.

\section{REQUERIMIENTOS}
\begin{tabular}{|c|c|p{2.8cm}|p{5.2cm}|p{2.8cm}|}
\hline
 \textbf{Ítem} & \textbf{Cant.} & \textbf{Hrs. Académicas} & \textbf{Nombre de la Auxiliatura} & \textbf{Código de la Auxiliatura} \\
\hline
 1 & 7 Aux. & 80 Hrs/mes & Administrador de Laboratorio de Cómputo & LCO-ADM \\
\hline
 2 & 3 Aux. & 80 Hrs/mes & Administrador de Laboratorio de Desarrollo & LDS-ADM \\
\hline
 3 & 2 Aux. & 56 Hrs/mes & Auxiliar de Laboratorio de Desarrollo & LDS-AUX \\
\hline
 4 & 1 Aux. & 80 Hrs/mes & Administrador de Laboratorio de Mantenimiento de Software & LM-ADM-SW \\
\hline
 5 & 4 Aux. & 32 Hrs/mes & Auxiliar de Laboratorio de Mantenimiento de Software y Hardware & LM-AUX-SW \\
\hline
 6 & 1 Aux. & 80 Hrs/mes & Administrador de Laboratorio de Mantenimiento de Hardware & LM-ADM-HW \\
\hline
 7 & 4 Aux. & 32 Hrs/mes & Auxiliar de Laboratorio de Mantenimiento de Hardware & LM-AUX-HW \\
\hline
 Total & 22 Aux. & \multicolumn{3}{|l|}{} \\
\hline
\end{tabular}

\section{REQUISITOS}
\begin{enumerate}[a)]
\item Ser estudiante de las carreras de Lic. o Ing. en Informática o Licenciatura en Ingeniería de Sistemas y/o afín, que cursa regularmente en la universidad.
\item O haber concluido el plan de estudios (ser egresado), pudiendo postular a la Auxiliatura Universitaria dentro del siguiente periodo académico (dos años o cuatro semestres), a partir de la fecha de conclusión de su plan de materias. Este periodo de dos años no podrán ampliarse bajo circunstancia alguna, aun en caso de encontrarse cursando otra carrera.
\item No tener deudas de libros en la biblioteca de la FCyT.
\item No tener deudas de libros en la biblioteca del Centro de Estudiantes de la carrera correspondiente.
\item Haber aprobado la totalidad de materias hasta quinto semestre.
\item Participar y aprobar el concurso de meritos y proceso de prueba de selección.
\end{enumerate}

NOTA: Se considera conclusión del plan de materias, el haber cursado el número de las materias requeridas en la curricula de la Carrera respectiva, exceptuando la materia titulación.

\section{DOCUMENTOS REQUISITOS A PRESENTAR}
\begin{enumerate}[a)]
\item Presentar Solicitud escrita para cada auxiliatura especificando claramente la auxiliatura a la que se postula: código de auxiliatura, nombre de la auxiliatura, dirigida a la Jefatura de Departamento.
\item Kardex actualizado gestión 2011, expedido por registros e inscripciones de la Facultad. (Oficina de Kardex en la Facultad).
\item Fotocopia del carnet de identidad.
\item Certificado de la Biblioteca de la FCyT donde se evidencia que no tiene pendiente deuda de libros prestados.
\item Presentar resumen de currículum Vitae de acuerdo a formato publicado en la página: \url{http://www.cs.umss.edu.bo}, sección convocatorias.
\item Presentar documentación que respalde el currículum vitae, ORGANIZADO Y SEPARADO de acuerdo a la tabla de calificación de meritos.
\end{enumerate}

NOTA: La documentación y las fotocopias de certificados, deben ser validadas gratuitamente en Secretaria del Departamento de Informática y Sistemas. (Presentar original y fotocopia). Se deberá presentar documentación separada por CADA UNA de las postulaciones, debidamente FOLIADA. La documentación no será devuelta.

\section{FECHA Y LUGAR DE PRESENTACIÓN DE DOCUMENTOS}
\subsection{DE LA FORMA}
Presentación de la documentación en sobre manila cerrado y rotulado con:

\begin{itemize}
\item Nombre y apellidos completos, dirección, teléfono(s) y e-mail del postulante.
\item Código de ítem de la auxiliatura a la que se postula.
\item Nombre de la auxiliatura a la que se presenta.
\end{itemize}

\subsection{DE LA FECHA Y LUGAR}
La documentación deberá ser presentada hasta horas 11:30 del día 5 de Julio del 2012, en Secretaria del Departamento con la Sra. Fabiola Rojas Caballero.

\section{CALIFICACIÓN DE MERITOS}
La calificación de méritos se basará en los documentos presentados por el postulante y se realizará sobre la base de 100 puntos que representa el 20\% del puntaje final y se pondera de la siguiente manera.

\begin{tabular}{|p{14cm}|c|}
\hline
 \textbf{RENDIMIENTO ACADÉMICO} & 65 \\
\hline
 Promedio general de las materias cursadas & 35 \\
\hline
 Promedio general de materias del anterior semestre & 30 \\
\hline
 \textbf{EXPERIENCIA GENERAL} & 35 \\
\hline
 \textbf{Documentos de experiencia en laboratorios} & 20 \\
\hline
 Auxiliar de Laboratorio Departamento de Informática - Sistemas del ítem respectivo: & 12 \\
\hline
 2 pts/semestre Auxiliar titular. &  \\
\hline
 1 pts/semestre Auxiliar Invitado. &  \\
\hline
 Auxiliares AdHonorem Laboratorio Departamento de Informática - Sistemas: & 6 \\
\hline
 1 pts/semestre Auxiliar. &  \\
\hline
 Otros auxiliares en laboratorios de computación: & 2 \\
\hline
 1 pts/semestre Auxiliar. &  \\
\hline
 \textbf{Producción} & 5 \\
\hline
 Disertación cursos y/o participación en Proyectos: & 5 \\
\hline
 2.5 pto/certificado &  \\
\hline
 \textbf{Documentos de experiencia extrauniversitaria y de capacitación} & 10 \\
\hline
 Experiencia como operador, programador, analista de sistemas, cargo directivo en centro de cómputo: & 6 \\
\hline
 2 puntos por certificado &  \\
\hline
 Certificación de capacitación en el área especifica expedidos por el sistema universitario & 4 \\
\hline
 2 ptos/certificado aprobación &  \\
\hline
 1 pto/certificado asistencia &  \\
\hline
\end{tabular}

NOTA: Todo certificado será ponderado hasta el valor del puntaje especificado en la tabla.

\section{CALIFICACION DE CONOCIMIENTOS}
La calificación de conocimientos se realiza sobre la base de 100 puntos, equivalentes al 80\% de la calificación final.

Las pruebas para los auxiliares sobre conocimientos se realizaran de acuerdo al temario y tabla siguiente.

\subsection{PORCENTAJES DE CALIFICACION PARA CADA TIPO DE AUXILIAR}
\subsubsection{PRUEBAS ESCRITAS}
Los postulantes deben de forma obligatoria rendir todas las pruebas escritas en el (los) ítem(es) a los que se postulan.

\begin{tabular}{|c|p{5.8cm}|p{0.8cm}|p{0.8cm}|p{0.8cm}|p{0.8cm}|p{0.8cm}|p{0.8cm}|p{0.8cm}|}
\hline
 \textbf{\#  } & \textbf{Temática} & \multicolumn{7}{|l|}{\textbf{Código de Auxiliatura}} \\
\hline
  &  & \textbf{\textbf{LCO-ADM}} & \textbf{LDS-ADM} & \textbf{LDS-AUX} & \textbf{LM-ADM-SW} & \textbf{LM-AUX-SW} & \textbf{LM-ADM-HW} & \textbf{LM-AUX-HW} \\
\hline
 1 & ADM LINUX BASICO - AVANZADO & 25 & 10 & 10 & 10 & 10 & 10 & 10 \\
\hline
 2 & REDES NIVEL INTERMEDIO & 25 &  &  & 10 & 10 &  &  \\
\hline
 3 & POSTGRES, MYSQL NIVEL INTERMEDIO & 20 & 20 & 30 &  &  &  &  \\
\hline
 4 & PROGRAMACION PARA INTERNET, LENGUAJES DE PROGRAMACION (JSP, JAVASCRIPT, CSS, HTML, PHP, DELPHI) &  & 40 & 40 &  &  &  &  \\
\hline
 5 & MODELAJE DE APLICACIONES WEB (UML),PROCESO UNIFICADO ESTRUCTURADO &  & 20 & 20 &  &  &  &  \\
\hline
 6 & ENSAMBLAJE Y MANTENIMIENTO DE COMPUTADORA EN HARDWARE Y SOFTWARE & 20 &  &  & 25 & 25 & 30 & 25 \\
\hline
 7 & ELECTRONICA APLICADA & \multicolumn{7}{|l|}{} \\
\hline
  & Teórico &  &  &  & 20 & 30 & 25 & 30 \\
\hline
  & Practico &  &  &  & 25 & 25 & 25 & 25 \\
\hline
 8 & DIDACTICA & 10 & 10 &  & 10 &  & 10 & 10 \\
\hline
\end{tabular}

\section{DE LOS TRIBUNALES}
Los Honorables Consejos de Carrera de Informática y Sistemas designarán respectivamente; para la calificación de méritos 1 docente y 1 delegado estudiante, de la misma manera para la comisión de conocimientos cada consejo designará 1 docente y un estudiante veedor por cada temática.

\section{CRONOGRAMA GENERAL}
\begin{tabular}{|l|l|}
\hline
 \textbf{Eventos} & \textbf{Fechas} \\
\hline
 Publicación de convocatoria. & 19 de Junio de 2012 \\
\hline
 Presentación de documentos. & 5 de Julio de 2012 \\
\hline
 Publicación de rol de pruebas practicas y de conocimientos. & 6 de Julio de 2012 \\
\hline
 Publicación de resultados. & 17 de Julio de 2012 \\
\hline
\end{tabular}

\section{SELECCIÓN}
Una vez concluido el proceso la jefatura decidirá qué auxiliares serán seleccionados para cada ítem, considerando los resultados finales y las necesidades propias de cada laboratorio.

\vspace{4cm}

\begin{center}
Cochabamba, 19 de Junio de 2012.
\end{center}

\vspace{4cm}

\begin{minipage}[b]{0.5\textwidth}
\begin{center}
{\bf Msc. Lic. Rolando Jaldin Rosales}\\
Dir. Carr. Informática\\
\end{center}
\end{minipage}
\begin{minipage}[b]{0.5\textwidth}
\begin{center}
{\bf Lic. Yony Montoya Burgos}\\
Dir. Carr. Ing. Sistemas\\
\end{center}
\end{minipage}

\vspace{4cm}

\begin{minipage}[b]{0.5\textwidth}
\begin{center}
{\bf Lic. Henrry Frank Villarroel Tapia}\\
Jefe Dpto. Informática-Sistemas\\
\end{center}
\end{minipage}
\begin{minipage}[b]{0.5\textwidth}
\begin{center}
{\bf Ing. Hernan Flores Garcia}\\
Decano FCyT-UMSS\\
\end{center}
\end{minipage}

\end{document}